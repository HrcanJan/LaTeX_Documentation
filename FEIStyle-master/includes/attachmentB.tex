Vitaj! Práve sa nachádzaš na našej webstránke „Interaktívne animácie IGBT tranzistora“. V tomto súbore sa nachádza manuál pre Teba ako s touto stránkou pracovať, ako aj minimálne špecifikácie, ktoré sú potrebné aby sa táto webstránka obrazovala korektne.

Dôležité je si uvedomiť, že táto stránka sa zobrazuje ako webová aplikácia, takže na jej prehliadanie je potrebné mať webový prehliadač. Pretože táto práca je vytvorená pomocou štandardu HTML5, tak animácia bude fungovať iba na nasledovných prehliadačoch: Google Chrome (verzia 52 a nahor), Mozilla Firefox (verzia 54 a nahor), Opera (verzia 39 a nahor), Safari (verzia 10.1 a nahor) a Microsoft Edge (verzia 14 a nahor). Na starších, alebo nespoľahlivých prehliadačoch, ako je Internet Explorer, táto aplikácia nebude funkčná. Avšak, aplikácia je responzívna, preto bude fungovať na všetkých zariadeniach, ako sú mobilné telefóny, desktop, tablety a iné. Na starších a slabších zariadeniach však môže mať problémy s plynulosťou zobrazenia. 

A ako postupovať, ak sa chceš dozvedieť viac? Prvým krokom je otvoriť linku na prehliadači, kde tá stránka bude umiestená, http://uef.fei.stuba.sk/moodleopen/course/view.php?id=104, 
alebo si ju otvoriť lokálne na počítači. Na počítači je potrebné otvoriť priečinok “IGBT Tranzistory” a v tom priečinku otvoriť súbor index.html. Ten súbor Ti otvorí hlavnú webstránku v lokálnom prehliadači. 

Po načítaní hlavnej stránky sa môžeš pozrieť na naše úvodné video tak, že stlačíš tlačidlo Úvod/Intro. Ďalej je možné otvoriť niektorú zo šiestich podstránok, kliknutím na Čítajte tu/Read more. Takisto, môžeš používať navigačné menu na prepínanie medzi stránkami. Kliknutím na logo našej stránky sa vrátiš späť na domovú stránku. Taktiež si môžeš meniť predvolený jazyk kliknutím na vlajky v hornom pravom kúte (máme dostupnú Slovenčinu a Angličtinu). 

Na podstránke \textbf{Princíp činnosti} nájdeš hlavnú animáciu IGBT. Na ľavej strane sa nachádza bočné menu s rôznymi možnosťami nastavenia animácie. Môžeš si zvoliť rôzne hodnoty napätia \textit{U}$_{GE}$ a \textit{U}$_{CE}$. Takisto môžeš animácie pozastaviť stlačením tlačidla "Pozastav animáciu". Opätovné stlačenie tohto tlačidla Ti umožní pokračovať v ukážke animácie. Ak budeš chcieť, môžeš skryť alebo zobraziť označenia v animácii IGBT tranzistora, ako aj skryť alebo znázorniť legendu. Posledným tlačidlom v ľavom stĺpci je Teória, kde je pomocou textu a obrázkov vysvetlené ako IGBT tranzistor funguje. 

V \textbf{historickom okienku} nájdeš krátku prezentáciu vývoja IGBT tranzistora. Môžeš sa medzi snímkami prepínať tak, že posúvaš guľôčku na slideri vľavo alebo vpravo, alebo môžeš použiť šípky na klávesnici.

Na \textbf{stránke aplikácie} uvidíš krátku prezentáciu niektorých aplikácii IGBT tranzistora. Stránka obsahuje krátku informáciu o tom, ako široko sa dá použiť IGBT tranzistor, a potom môžeš posúvať slideri aby si videl iné snímky prezentácie. Ak si budeš meniť snímku na obrazovke, tak sa mení druhá časť popisu, ktorá popisuje akú úlohu tranzistor má v danom príklade.

Na stránke \textbf{Výhody / Nevýhody} nájdeš dva jednoduché a krátke zoznamy, ktoré popisujú výhody, respektíve nevýhody IGBT tranzistora.

Stránka \textbf{Teória IGBT} uvádza základné informácie  o tom, čo sa deje v IGBT tranzistore, aby sa IGBT naštartoval. Môžeš znovu použiť slider na presúvanie sa medzi jednotlivými snímkami, pričom každá snímka má svoj vlastný popis. V dolnom ľavom rohu sa nachádza tlačidlo, ktoré Ťa presunie naspäť k hlavnej animácii.

Ak si chceš zistiť, čo si sa naučil stlač tlačidlo FAQ. Na stránke \textbf{Frequently Asked Questions} nájdeš otázky týkajúce sa IGBT. Ak sa rozhodneš položiť novú otázku, tak táto otázka bude odoslaná na univerzitný mail a budeme sa snažiť na ňu odpovedať. Akonáhle prečítame prijatú otázku, aktualizujeme stránku s odpoveďou na ňu.

Prajeme Ti veľa úspechov pri študovaní našej animácie.