V dnešnej dobe sa vo veľkom množstve elektronických výkonnových zariadení používa tranzistor s izolovaným hradlom (\acrshort{IGBT}). IGBT sa vyskytuje v priemyselnej technológii, v bežných elektronických prístrojoch (akými sú mikrovlnná rúra, chladničky a klimatizačné zariadenia), v energetickom sektore, vesmírnych zariadeniach aj v nových dopravných prostriedkoch \cite{c12}.
\newline IGBT tranzistory sú krokom do budúcnosti vo výkonovej elektronike a preto je dôležité pochopiť princíp takejto súčiastke, čo však nie je ľahké. Obyčajný informačný text o IGBT sa môže zdať nezaujímavý ale kombinovaním animácie s interaktivitou sa tento problém môže riešiť. Hlavná výhoda animácií je vizualizácia a zjednodušenie zložitej teórie. Interaktivita môže u študentov vyvolať  záujem o vzdelávanie a o zveľadenie vlastných vedomostí. Aby sme sa k tomu dopracovali, potrebné je analyzovať ako funguje animácia, ako využiť interaktívne elementy v našej práci a vykonať analýzu softvérov, ktoré možno použiť na implementáciu nášho návrhu. V nasledujúcom kroku bude dôležité plánovať rozloženie našich animácií na webovej stránke a plánovať formát tejto stránky. Nakoniec,  musíme sa z teoretickej časti dostať k praktickej, tiež  vytvoriť a nahrať naše animácie na web.
\newline Na Fakulte elektrotechniky a informatiky v Bratislave sú už od roku 2001 vyvíjané elearning materiály, súčasťou ktorých sú interaktívne animácie. Tieto vzdelávacie materiály boli voľne sprístupnené študujúcim v roku 2004 v rámci vzdelávacieho portálu eLearn central, pracujúceho na platforme LMS Moodle a stále sa využívajú na vzdelávanie študentov na našej fakulte. Vytvorené e-learning moduly však po určitom čase a pri vývoji nových technológii zastarajú. Adobe Flash prestál byť podporovaný a tým pádom animácie nie sú funkčné. Preto je nevyhnutné aktualizovať staré a vytvoriť nové aplikácie pre našich študentov \cite{c13}.
\newline Cieľom tejto práce je vývoj interaktívnych animácií IGBT tranzistora pre študentov  vysokých a stredných škôl. Chceme ich povzbudiť a vyvolať u nich  záujem o pochopenie princípu činnosti IGBT tranzistorov, predovšetkým zábavným a pre nich zrozumiteľným spôsobom.