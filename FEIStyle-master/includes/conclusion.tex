IGBT sú kľúčové komponenty výkonových elektronických systémov, ktoré sú mimoriadne dôležité v aplikáciách, kde potrebujeme spínať veľké prúdy a napätia. 
Táto bakalárska práca sa zaoberá návrhom a vývojom interaktívnej animácie IGBT tranzistora. V časti analýzy sme stručne uviedli význam interaktívnej vzdelávacej animácie; prehľad softéru na programovanie týchto animácii a následne sme si stručne vysvetlili princíp činnosti a použitia IGBT tranzistora. Na základe výsledkov analýzy sme pripravili riešenie vrátane grafického a logického návrhu. V rámci praktickej časti práce sme vytvorili a implementovali animáciu. Ďalej sme overili naše riešenie a opísali technické problémy. Na záver sme vytvorili technickú dokumentáciu na vysvetlenie funkčnosti projektu na počítači.

Výstupom tohto projektu je interaktívna animácia IGBT tranzistora zobrazujúca päť rôznych častí: princíp činnosti, historické okienko, použitie IGBT tranzistora, výhody/nevýhody IGBT tranzistora a popis jeho hlavnej súčiastky. Toto usporiadanie pomáha študentom hlbšie pochopiť princíp IGBT tranzistora.

Naša animácia bude dostupná na vzdelávacom portáli eLearn Central Ústavu elektroniky a fotoniky s voľným prístupom pre študentov STU alebo ostatných návštevníkov stránky.