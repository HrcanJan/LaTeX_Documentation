Táto bakalárska práca sa zaoberá návrhom a vývojom interaktívnej animácie bipolárneho tranzistora s izolovaným hradlom (\acrshort{IGBT}). V časti analýzy je uvedený základný význam interaktívnej vzdelávacej animácie; na programovanie sme si vybrali softvér a následne sme si stručne vysvetlili princíp činnosti a použitia \acrshort{IGBT} tranzistora. Na základe výsledkov analýzy sme pripravili riešenie, vrátane grafického a logického návrhu. Následuje popis návrhu tvorby a implementácie samotnej animácie. V ďaľšej časti práce sme overili naše riešenie a opísali technické problémy. Na záver sme spravili technickú dokumentáciu na vysvetlenie funkčnosti projektu na počítači. Výstupom tohto projektu je interaktívna animácia \acrshort{IGBT} tranzistora, zobrazujúca päť rôznych častí: princíp činnosti, historické okienko, použitie \acrshort{IGBT} tranzistora, výhody/nevýhody \acrshort{IGBT} tranzistora a teóriu IGBT. Toto usporiadanie pomáha študentom hlbšie pochopiť ako pracuje \acrshort{IGBT}. Naša animácia bude dostupná na vzdelávacom portáli eLearn central Ústavu elektroniky a fotoniky s voľným prístupom pre študentov STU ako aj ostatných návštevníkov stránky.